\newacronym{lul}{LuL}{Lehrer*innen}
%\newglossaryentry{code}{
%	name={name},
%	description={\enquote{Erklärung}}
%   \cite{optional wiki:línk}
%}

\newglossaryentry{bootstick}{
	name={Bootstick},
	description={\enquote{\textit{Bootstick} ist die projektinterne Bezeichnung für jeden USB-Stick, welcher das Betriebssystem des Produktes speichert und bereitstellt.}}
}
\newglossaryentry{datenstick}{
	name={Datenstick},
	description={\enquote{\textit{Datenstick} ist die projektinterne Bezeichnung für jeden USB-Stick, welcher die Angaben des Lehrpersonals und die Abgaben der Schülerschaft in einer Leistungsüberprüfung speichert.}}
}
\newglossaryentry{stickpaar}{
	name={Stickpaar},
	description={\enquote{\textit{Stickpaar} ist die projektinterne Bezeichnung für einen \gls{bootstick} und einen \gls{datenstick}, welche zusammen zu verwenden sind. Sie beinhalten passende Schlüssel für diverse verschlüsselte Daten auf dem jeweils anderem Objekt und überprüfen selbstständig, ob das korrekte Gegenstück in Verwendung ist bzw. versuchen die Leistungsüberprüfung der Schülerschaft inkorrekten Gegenstücken zu erschweren.}}
}
\newglossaryentry{nfm}{
	name={NFM},
	description={\enquote{Der \textit{\textbf{N}icht-\textbf{F}unktionale-\textbf{M}odus} ist ein projektinterner Begriff für einen Zustand, in welchem nur das Abschalten bzw. Neustarten des Systems möglich ist.}}
}
%Dieser unübliche Begriff wurde gewählt, um ein mögliches Ärgernis der Lehrerschaft und/oder des Ministeriums über den Begriff "Kernel Panic" und dessen Bedeutung zu vermeiden.
\newglossaryentry{amt}{
	name={Intel® AMT},
	description={\enquote{Die \textit{Intel® \textbf{A}ctive \textbf{M}anagement \textbf{T}echnology} (Englisch für \textit{Aktive Verwaltungstechnologie}) ist ein von Intel® entwickeltes, in Computer verbautes, System, um die Fernsteuerung des Rechners zu ermöglichen.}
	\cite{wiki:amt}}
}
\newglossaryentry{linux}{
	name={Linux},
	description={\enquote{Als \textit{GNU/Linux} (oft auch nur \textit{Linux}) wird eine Sammlung an freien Betriebssystemen bezeichnet, welche auf dem Linux-Kernel basieren.}
	\cite{wiki:linux}}
}
\newglossaryentry{admin}{
	name={Administration},
	description={\enquote{Ein/e \textit{Administrator/in} beschreibt projektintern eine Lehrperson, welche in der Verwendung des Produkts geschult wurde.}}
}
\newglossaryentry{deploy}{
	name={Deployment-Umgebung},
	description={\enquote{Die \textit{Deployment-Umgebung} beschreibt in diesem Projekt einen vom Auftragnehmer bereitgestellten Rechner, welcher für die Erstellung und Bearbeitung der \gls{stickpaar}e verwendet wird.}}
}



%Nützliche Definitionen
\newglossaryentry{timst}{
	name={Timestamp},
	description={\enquote{Ein Zeitstempel (englisch timestamp) wird benutzt, um einem Ereignis einen eindeutigen Zeitpunkt zuzuordnen.} \cite{https://de.wikipedia.org/wiki/Zeitstempel}}
}
\newglossaryentry{pdf}{
	name={PDF},
	description={\enquote{PDF ist ein universelles Dateiformat, das besonders für das elektronische Publizieren und in der Druckvorstufe eingesetzt wird}}
}
\newglossaryentry{sga}{
	name={Schulgemeinschaftsausschuss},
	description={\enquote{Der Schulgemeinschaftsausschuss (SGA) ist ein gesetzlich verankertes Gremium für mittlere und höhere Schulen in Österreich.}}
}
\newglossaryentry{speicherstick}{
	name={Speicherstick},
	description={\enquote{Ist ein dummer Stick}}
}
\newglossaryentry{webinterface}{
	name={Webinterface},
	description={\enquote{Ein Web Interface ist ein System, durch welches Anwender mit dem Netz interagieren. Der Begriff Web Interface steht zumeist für grafische Oberflächen.}}
}