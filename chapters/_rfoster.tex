\section{Frontend - Webapplikation als REST-Client}
\subsection{Überblick}
Das \Gls{backend} muss mit dem \Gls{frontend} verbunden werden.Es gibt unterschiedliche Möglichkeiten dies zu realisieren. Beim Realisieren, muss darauf geachtet werden, dass eine Struktur vorhanden ist. Es werden 2 verschiedene \Gls{design-pattern} angeschaut und verglichen. Umgesetzt wird dann ein \Gls{design-pattern} mithilfe von \Gls{js}. Hier kann ein \Gls{jsframework} zum Einsatz kommen. Dazu werden hier verschiedene \Gls{jsframework}s angeschaut und verglichen. Für die Verarbeitung der Daten ist es wichtig Datenformate festzulegen. Die Aufbereitung der Elemente fürs \Gls{frontend} mit den Daten des \Gls{backend}s wird ebenfalls angeschaut.
\subsection{Design-Patterns}
\subsubsection{MVVM}
\subsubsection{MVC}
\subsubsection{Vergleich}
\subsection{Datenformate}
\subsection{Umsetzungsmöglichkeiten}
\subsubsection{Vue}
\subsubsection{React}
\subsubsection{Angular}
\subsubsection{Ohne Framework}
\subsubsection{Vergleich}
\subsection{Aufbereitung der Daten}
\subsection{Fragestellung}